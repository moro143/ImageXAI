% !TEX encoding = UTF-8 Unicode 
% !TEX root = praca.tex

\chapter*{Wstęp}

\section*{Wprowadzenie do problemu}
Klasyfikacja obazów przy użyciu sieci głębokich (DNN) stanowi jeden z najważniejszych obszarów badawczych w dziedzinie sztucznej inteligencji.
Dzięki swoim niezwykłym możliwościom w rozpoznawaniu wzorców, sieci głębokie są szeroko stosowane w różnych aplikacjach, takich jak diagnostyka medyczna, autonomiczne pojazdy, czy systemy rozpoznawania twarzy.
Pomimo ich imponujących wyników, sieci głębokie są często krytykowane za swoją "czarną skrzynkę" - trudność w zrozumieniu, jak dokładnie podejmują one decyzje.
W odpowiedzi na to wyzwanie, powstała dziedzina wyjaśnialnej sztucznej inteligencji (XAI), która ma na celu uczynienie decyzji podejmowanych przez modele AI bardziej przejrzystymi i zrozumiałymi dla użytkowników.

Potrzeba wyjaśniania decyzji podejmowanych przez modele DNN wynika z kilku kluczowych powodów.
Po pierwsze, zrozumienie, dlaczego model podjął konkretną decyzję, jest niezbędne dla zaufania użytkowników do technologii AI, zwłaszcza w krytycznych aplikacjach, gdzie błędy mogą mieć poważne konsekwencje.
Po drugie, wyjaśnienia pomagają w identyfikacji potencjalnych błędów modelu, umożliwiając jego dalszą optymalizację i ulepszanie.
Wreszcie, XAI może wspierać proces zgodności z regulacjami prawnymi które częściej wymagają przejrzystości algorytmicznej.

\section*{Cel pracy}
Celem niniejszej pracy jest zbadanie i porównanie różnych metod XAI stosowanych w klasyfikacji obrazów.
W ramach pracy zostaną przeanalizowane istniejące techniki wyjaśniania decyzji podejmowanych przez modele głębokie, a także ocenione ich skuteczność.
Kluczowym elementem pracy będzie porównanie wybranych metod XAI pod kątem ich zdolności do generowania zrozumiałych i wiarygodnych wyjaśnień.

\section*{Zakres pracy}
Zakres pracy obejmuje kilka kluczowych aspektów.
Po pierwsze, przygotowane zostanie środowisko programistyczne oraz odpowiedni zbiór danych, który umożliwi przeprowadzenie eksperymentów.
W tym celu zostaną wybrane i skonfigurowane narzędzia oraz biblioteki niezbędne do implementacji i testowania metod XAI.
Kolejnym krokiem będzie praktyczne badanie wybranych metod XAI.
Będą one stosowane do analizowania wyników uzyskanych przez modele DNN na przygotowanym zbiorze danych.
Na koniec, wyniki eksperymentów zostaną porównane i omówione, co pozwoli na wyciągnięcie wniosków dotyczących efektywności poszczególnych metod oraz wskazanie ich zalet oraz ograniczeń.
