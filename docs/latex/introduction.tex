% !TEX encoding = UTF-8 Unicode 
% !TEX root = praca.tex

\chapter*{Wstęp}

\section*{Wprowadzenie do problemu}
Klasyfikacja obazów przy użyciu sieci głębokich (Deep Neral Networks, DNN) stanowi jeden z najważniejszych obszarów badawczych w dziedzinie sztucznej inteligencji.
Dzięki swoim niezwykłym możliwościom w rozpoznawaniu wzorców, sieci głębokie są szeroko stosowane w różnych aplikacjach, takich jak diagnostyka medyczna, autonomiczne pojazdy, czy systemy rozpoznawania twarzy.
Te modele uczą się z ogromnych ilości danych, co pozwala im na osiąganie wysokiej precyzji w zadaniach klasyfikacyjnych.

Pomimo ich imponujących wyników, sieci głębokie są często postrzegane jako "czarna skrzynka".
Oznacza to, że procesy decyzyjne zachodzące wewnątrz tych modeli są trudne do zrozumienia, nawet dla ekspertów w dziedzinie.
Brak przejrzystości w działaniu modeli DNN jest istotnym problemem, zwłaszcza w kontekście zastosowań wymagających wysokiego poziomu zaufania i odpowiedzialności, takich jak opieka zdrowotna czy systemy bezpieczeństwa.

Wyjaśnialna sztuczna inteligencja (Explainable AI, XAI) powstała w odpowiedzi na ten problem.
XAI ma na celu uczynienie decyzji podejmowanych przez modele AI bardziej przejrzystymi i zrozumiałymi dla użytkowników.
Potrzeba wyjaśniania decyzji podejmowanych przez modele DNN wynika z kilku kluczowych powodów:
\begin{itemize}
  \item \textbf{Zaufanie i akceptacja} - zrozumienie, dlaczego model podjął konkretną decyzję, jest niezbędne dla budowania zaufania użytkowników do technologii AI. 
    Jest to szczególnie ważne w aplikacjach krytycznych, gdzie błędy mogą prowadzić do poważnych konsekwencji. 
  \item \textbf{Optymalizacja i poprawa modelu} - Wyjaśnienia pomagają w identyfikacji potencjalnych błędów modelu, co umożliwia jego dalszą optymalizację i ulepszanie.
  \item \textbf{Ocena modelu} - wyjaśnienia mogą służyć jako alternatywne miary oceny modelu, pozwalając na bardziej wszechstronne zrozumienie jego działania w kontekście rzeczywistych danych.
  \item \textbf{Zgodnośc z regulacjami} - w wielu branżach istnieją regulacje wymagające przejrzystości, co sprawia, że wyjaśnialność decyzji modeli AI jest konieczna.
\end{itemize}

W kontekście klasyfikacji obrazów, metody XAI pomagają zrozumieć, które cechy obrazu miały największy wpływ na decyzję modelu.
Pozwala to na lepszą interpretację wyników i identyfikację potencjalnych obszarów do poprawy modelu.

\section*{Cel pracy}
Celem niniejszej pracy było zbadanie i porównanie różnych metod XAI stosowanych w klasyfikacji obrazów.
W ramach pracy przeanalizowano istniejące techniki wyjaśniania decyzji podejmowanych przez modele głębokie oraz oceniono ich skuteczność w kontekście klasyfikacji obrazów.
Kluczowym elementem pracy było porównanie wybranych metod XAI pod kątem ich zdolości do generowania zrozumiałych i tranych wyjaśnień, które wskazują jakie cechy obrazu miały największy wpływ na decyzję modelu.

Praca miała na celu:
\begin{itemize}
  \item \textbf{Zidentyfikowanie i zbadanie popularnych metod XAI stosowanych w klasyfikacji obrazów}
  \item \textbf{Ocenę skuteczności i użyteczności tych metod}
  \item \textbf{Porównanie metod XAI}
  \item \textbf{Identyikacja zalet i ograniczeń poszczególnych metod}
\end{itemize}

Efektem końcowym pracy było lepsze zrozumienie, które metody XAI są najbardziej efektywne i jakie są ich praktyczne zastosowania w kontekście klasyfikacji obrazów.

\section*{Zakres pracy}
Zakres pracy obejmował kilka kluczowych aspektów, mających na celu kompleksowe zbadanie i porównanie metod wyjaśnialnej sztucznej inteligencji (XAI) stosowanych w klasyfikacji obrazów.
Praca została podzielona na następujące aspekty:
\begin{itemize}
  \item \textbf{Przegląd literatury} w celu zapoznania się z metodami klasyfikacji orazów, metodami XAI oraz podobnymi pracami badawczymi, które umożliwiły zrozumienie obecnego stanu wiedzy w tej dziedzinie. 
  \item \textbf{Analiza i wybór odpowiednich parametrów} dla metod XAI oraz ustalenie sposobów przekształcania i prezentacji wyjaśnień generowanych przez metody, aby były one porównywalne.
  \item \textbf{Stworzenie odpowiedniego środowiska programistycznego} oraz \textbf{zgromadzenie zbiioru danych} który umożliwił przeprowadzenie badań. 
  \item \textbf{Implementacja i zastosowanie wybranych metod XAI} w klasyfikacji obrazów oraz \textbf{analiza wyników}, koncentrując się na ocenie, jak te metody wyjaśniają decyzje modeli głębokiego uczenia i jakie informacje dostarczają użytkownikom.
  \item \textbf{Porównanie wyników} poprzez ocenę efektywności poszczegółnych metod XAI oraz \textbf{wyciągnięcie wniosków} dotyczących ich skuteczności oraz identyfikacja zalet i ograniczeń każdej z metod.
\end{itemize}

Praca miała na celu kompleksową analizę i porównanie metod XAI, co miało prowadzić do lepszego zrozumienia ich praktycznych zastosowań oraz wskazanie kierunków dla przyszłych badań w tej dziedzinie.
