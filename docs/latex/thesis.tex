% !TEX encoding = UTF-8 Unicode 
%
% Use:
% magister / inzynier - for master thesis or engineering thesis
% druk / archiwum - for print version or archive version
% en - to translate template into english
% examples:
%\documentclass[inzynier,druk,en] - master thesis, print version, english
%\documentclass[magister,druk,en]{dyplom}
%\documentclass[magister,druk]{dyplom}

\documentclass[magister,druk]{dyplom}

\usepackage[utf8]{inputenc}
\usepackage{hyperref}

% Maximum section's depth.
\setcounter{secnumdepth}{4}

% Listings settings
\setminted{breaklines, 
frame=lines,           
framesep=3mm,          
baselinestretch=1.1,   
fontsize=\small,       
% linenos              % line numbering
}

\usepackage{lipsum}

% \faculty{Faculty of \dots}                   % Uncomment if applicable
\fieldofstudy{Informatyka techniczna}                          
\author{inż. Mateusz Moroszczuk}
\title{Porównanie metod XAI w klasyfikacji obrazów}
\supervisor{dr hab. inż. Henryk Maciejewski}
% \consultant{Consultant's name}               % Uncomment if applicable
% \specialisation{AAA}                         % Uncomment if applicable
\keywords{XAI, klasyfikacja obrazów}	% 3-5 keywords  

\begin{document}

\maketitle

\abstract{
% English abstract 


}{
% Abstract translated into Polish


}


\tableofcontents

% Wprowadzenie [Prezentacja tematu i cel pracy, Znaczenie XAI w zadaniu klasyfikacji obrazów, Wybrane metody XAI]
% !TEX encoding = UTF-8 Unicode 
% !TEX root = praca.tex

\chapter*{Wprowadznie}

\section*{Wprowadzenie do problemu}
Przedstawienie problemu klasyfikacji obrazów przy użyciu sieci głębokich.

Potrzeba XAI do zrozumienia podejmowanych decyzji.

\section*{Cel pracy}
Przedstawienie celu: zbadanie i porównanie metod XAI.
O zbiorze ImageNet, skupić się na klasach obrazów charakteryzujących się niską 'robustness'

\section*{Zakres pracy}
Przygotowanie środowiska programistycznego i zbioru danych.
Analiza lieratury
Badania różnych metod XAI

\section*{Wybrane Metody XAI}

W dziedzinie wyjaśnialnej sztucznej inteligencji (XAI) istnieje wiele metod służących do wyjaśnienia decyzji podejmowanych przez modele uczenia maszynowego, zwłaszcza sieci neuronowe.
W naszej pracy skupiamy się na trzech głównych metodach XAI, które są szeroko stosowane i dobrze zbadane:

\begin{enumerate}
	\item \textbf{LIME (Local Interpretable Model-agnostic Explanations)}: LIME jest techniką stosowaną do generowania lokalnych interpretacji modeli uczenia maszynowego, które są agnostyczne względem modelu.
	      Metoda ta polega na generowaniu lokalnych wyjaśnień dla indywidualnych predykcji modelu, co pozwala na zrozumienie, dlaczego model dokonał konkretnej klasyfikacji dla danego przypadku testowego.

	\item \textbf{SHAP (SHapley Additive exPlanations)}: SHAP to metoda oparta na teorii gier, która dostarcza globalnych interpretacji modeli uczenia maszynowego.
	      Wykorzystuje ona wartości Shapleya, aby obliczyć wpływ każdej cechy na predykcje modelu.
	      SHAP umożliwia zrozumienie, jak poszczególne cechy przyczyniają się do wyników modelu na poziomie globalnym.

	\item \textbf{GradCAM (Gradient-weighted Class Activation Mapping)}: GradCAM to technika wizualizacji, która pozwala na lokalizowanie istotnych obszarów na obrazie, które przyczyniają się do konkretnej predykcji modelu.
	      Wykorzystuje ona gradienty ostatniej warstwy sieci neuronowej w celu generowania map aktywacji klas, co umożliwia zrozumienie, które obszary obrazu były najistotniejsze dla decyzji modelu.
\end{enumerate}



% Przegląd literatury
% !TEX encoding = UTF-8 Unicode 
% !TEX root = praca.tex

\chapter*{Przegląd literatury}
\section*{Metody klasyfikacji obrazów}

Klasyfikacja obrazów to jedno z fundamentalnych zagadnień w dziedzinie sztucznej inteligencji, które ma na celu przypisanie etykiet klas do obrazów na podstawie ich cech.
Jest to zagadnienie o szerokim zastosowaniu, znajdujące praktyczne zastosowanie w wielu dziedzinach życia, od medycyny\cite{medical} i biologii\cite{biology} po przemysł, marketing\cite{marketing} czy bezpieczeństwo\cite{security}.

Jednym z głównych wyzwań w klasyfikacji obrazów\cite{imageclassificationchallanges} jest różnorodność danych i złożoność struktury obrazów.
Obrazy mogą zawierać wiele różnych obiektów, o różnych kształtach, rozmiarach, orientacjach i położeniach, co utrudnia automatyczne rozpoznawanie i klasyfikację.
Ponadto, obrazy mogą być podatne na różne rodzaje zniekształceń, takie jak zmiany oświetlenia, rozmazania czy zakłócenia, co dodatkowo utrudnia proces klasyfikacji.
Negatywny wpływ na wynik klasyfikacji mogą mieć niskiej jakości zbiory danych

Źródłami  problemów mogą być zbiory danych, w których wszystkie dane pochodzą z tego samego lub podobnego źródła, co może skutkować przeuczeniem się modelu, poprzez nauczenie się modelu na nieoczekiwanych cech w celu klasyfikacji.

\textbf{Głębokie sieci neuronowe (DNN)}\cite{nature} to modele, które składają się z wielu warstw neuronów, umożliwiając automatyczne uczenie się hierarchicznych cech na różnych poziomach abstrakcji, co sprawia, że są one bardzo skuteczne w analizie i rozpoznawaniu wzorców w danych obrazowych.

Istnieje wiele różnych architektur głębokich sieci neuronowych.
Przykładem są \textbf{sieci konwolucyjne (CNN)}\cite{8379889}, które zostały stworzone do przetwarzania obrazów.
Ich architektura opiera się na kilku podstawowych rodzajach warstw, które skutecznie wyodrębniają cechy obrazów na różnych poziomach abstrakcji.
Podstawowymi elementami sieci konwolucyjnych są:
\begin{itemize}
	\item \textbf{Warstwy konwolucyjne}, wykorzystują zestaw filtrów, które przesuwają się po obrazie, wykonując operację konwolucji.
	      Ta operacja pozwala na wyodrębnienie lokalnych cech z tego obrazu, takich jak krawędzie, tekstury czy wzory.
	      Tworząc nowe obrazy zwane mapami konwolucji.
	\item \textbf{Warstwy poolingowe}, występują po warstwach konwolucyjnych , redukują wymiary przestrzenne map cech.
\end{itemize}
Sieci konwolucyjne są szczególnie skuteczne w analizie danych obrazowych ze względu na ich zdolność do automatycznego wyodrębniania cech lokalnych i hierarchicznego uczenia się abstrakcji na różnych poziomach.
Ich zastosowanie znalazło szerokie zastosowanie ww rozpoznawaniu obrazów, klasyfikacji, detekcji i segmentacji obrazów, a także w innych dziedzinach przetwarzania danych, takich jak analiza tekstu czy przetwarzanie języka naturalnego.

\textbf{ResNet (Residual Neural Network)}\cite{He_2016_CVPR} to innowacyjna architektura głębokiej sieci neuronowej, która została zaprojektowana w celu rozwiązania problemu zanikającego gradientu w głębokich sieciach.
Problem zanikającego gradientu występuje gdy uczenie się modelu staje się trudniejsze wraz ze wzrostem liczby warstw, co prowadzi do ograniczenia wydajności modelu.

Główną cechą ResNet są tzw. bloki rezydualne, które wprowadzają połączenia skrótowe między warstwami, pozwalając na przekazywanie informacji wstecz w sieci.
Dzięki tym połączeniom model może uczyć się reszty funkcji, co eliminuje problem zanikającego gradientu.
ResNet znany jest ze swojej zdolności do efektywnego uczenia się bardzo głębokich modeli, nawet sięgających kilkuset warstw.
Ta cecha sprawia, ze ResNet stał się jedną z najbardziej popularnych architektur w dziedzinie rozpoznawania obrazów, osiągając znakomite wyniki.

\textbf{ResNet50}\cite{Koonce2021} jest konkretną wersją architektury ResNet, która składa się z 50 warstw.
Jest to jedna z najpopularniejszych wariantów ResNet często używaną podczas porównywania wyników, jako punkt odniesienia.

%\section*{Zastosowania klasyfikacji obrazów w różnych dziedzinach}

\section*{Metody XAI}
W ostatnich latach głębokie sieci neuronowe (DNN) osiągnęły niezwykłe sukcesy w wielu dziedzinach, takich jak rozpoznawanie obrazów, analiza tekstu czy przetwarzanie języka naturalnego.
Pomimo ich imponującej dokładności predykcji, istnieje coraz większa potrzeba zrozumienia, dlaczego modele te podejmują konkretne decyzje.
W szczególności w przypadku krytycznych zastosowań, takich jak medycyna czy bezpieczeństwo, zrozumienie powodów, dla których model dokonuje konkretnej predykcji, jest niezwykle istotne.

Wyjaśnialna sztuczna inteligencja (XAI)\cite{XAItax, XAIcurrent, XAIOnC, XAIcounter} ma na celu zwiększenie transparentności i zrozumienia procesu podejmowania decyzji przez modele głębokiego uczenia.
Posiadając wiedzę na temat danych treningowych oraz będąc w stanie określić jakie cechy miały wpływ na predykcję, można ocenić czy nauczył się w odpowiedni sposób.
Metody XAI pozwalają na analizę i interpretację działania modeli, co umożliwia użytkownikom oraz programistom zrozumienie, dlaczego model dokonał konkretnej predykcji i które cechy danych miały największy wpływ na tę decyzję.

Istnieją różnorodne metody wyjaśnialnej sztucznej inteligencji, można je grupować na wiele różnych sposobów w celu lepszego zrozumienia ich działania.
Jednym z nich to podział na wyjaśnienia lokalne\cite{ribeiro2016why} i wyjaśnienia globalne\cite{XAIglobal}.
Wyjaśnienia globalne przedstawiają działanie całego model, natomiast wyjaśnienia lokalne skupiają się na wyjaśnianiu decyzji dla poszczególnych instancji danych.
W problemie klasyfikacji danych używane są zazwyczaj lokalne wyjaśnienia.

Innym sposobem na grupowania XAI jest podział na dwie grupy:
\begin{itemize}
	\item \textbf{Metody niezależne od modelu} (Model-agnostic)
	\item \textbf{Metody zależne od modelu} (Model-specific)
\end{itemize}
Metody te różnią się swoim podejściem do wyznaczania wyjaśnień decyzji modeli i mają różne zalety oraz ograniczenia.

Poznanie zalet i ograniczeń tych metod, pomaga w lepszym zrozumieć, jakie są różnice między nimi, jak działają i w jaki sposób mogą być stosowane w praktyce.
Można dzięki temu wybrać odpowiednią metodę XAI dla konkretnych zastosowań oraz zwiększając tym zaufanie do modeli głębokiego uczenia poprzez ich lepsze zrozumienie i interpretację.

\subsection*{Metody niezależne od modelu}
Metody oparte na podejściu Model-agnostic są technikami XAI, które niezależnie od konkretnego modelu potrafią wyjaśnić jego decyzje.
Charakteryzują się one uniwersalnością i zdolnością do stosowania wobec różnych rodzajów modeli, co czyni je atrakcyjnym narzędziem dla praktyków i badaczy.

Główną zaletą tego podejścia jest jego niezależność od wewnętrznych mechanizmów modelu, co oznacza, że metody te mogą być stosowane do wyjaśniania decyzji zarówno prostych, jak i bardziej skomplikowanych modeli, w tym sieci neuronowych czy drzew decyzyjnych.

Przykładem takiej techniki jest \textbf{LIME} (Local Interpretable Model-agnostic Explanations)\cite{ribeiro2016why, LIMEwhy}, jedna z najpopularniejszych technik XAI, która umożliwia tworzenie lokalnie interpretowalnych modeli wokół wybranych przypadków danych.
Metoda ta polega na generowaniu lokalnych wyjaśnień, które opisują, jakie cechy danych wejściowych wpływają na decyzje modelu.
Poprzez symulowanie zbliżonych instancji danych i analizę reakcji modelu na ich zmiany, LIME pozwala na zrozumienie, które cechy mają największy wpływ na wynik predykcji.

Metoda LIME składa się z następujących kroków:
\begin{enumerate}
	\item \textbf{Wybór instancji do wyjaśnienia} - Na początku wybierana jest konkretna instancja danych, której decyzja modelu zostanie  wyjaśniona.
	\item \textbf{Generowanie perturbacji} - Następnie generowany jest zestaw perturbowanych wersji oryginalnej instancji.
	      Perturbacje te są tworzone przez wprowadzenie małych, losowych zmian do cech wejściowych.
	      W przypadku danych obrazowych, perturbacje mogą polegać na zmianie wartości super-pikseli.
	\item \textbf{Ocena perturbowanych instancji} - Każda perturbowana instancja jest następnie przekazywana do oryginalnego model, aby uzyskać predykcję dla nich
	\item \textbf{Ważenie perturbacji} - Perturbacje są ważone na podstawie ich odległości od oryginalnej instancji.
	      Perturbacje bliższe oryginalnej instancji mają większy wpływ na trenowanie lokalnego modelu.
	\item \textbf{Tworzenie lokalnego modelu} - LIME buduje prosty, lokalny model interpretowalny w oparciu o perturbowane instancje i ich odpowiadające predykcje, z uwzględnieniem ich wag.
	      Ten model jest trenowany, aby najlepiej odwzorować zachowanie oryginalnego modelu w okolicy wybranej instancji.
	\item \textbf{Generowanie wyjaśnień} - Po zbudowaniu lokalnego modelu interpretowalnego, LIME identyfikuje najważniejsze cechy, które wpływają na predykcję oryginalnego modelu dla wybranej instancji.
	      Wyjaśnienia te są przedstawiane w postaci listy cech z przypisanymi wagami, które pokazują ich wpływ na decyzję modelu.
\end{enumerate}

Poniżej znajduje się pseudokod prezentujący kroki algorytmu LIME:
\begin{listing}
	\begin{minted}{python}
        def explain_with_lime(instance):
            perturbed_instances = generate_perturbations(instance)
            predictions = model.predict(perturbed_instances)
            weights = compute_weights(perturbed_instances, instance)
            local_model = train_local_model(perturbed_instances, predictions, weights)
            explanations = generate_explanations(local_model)
            return explanations
  \end{minted}
	\caption{Pseudo kod LIME} \label{listing:lime}
\end{listing}

LIME jest techniką XAI, która pozwala na zrozumienie, jakie cechy danych wpływają na decyzje modelu, konkretnych instancji danych.
Dzięki LIME, użytkownicy mogą lepiej zrozumieć, dlaczego model podjął konkretną decyzję.
W przypadku danych obrazowych LIME umożliwia wizualizację wpływu poszczególnych fragmentów obrazu na decyzje modelu, co dodatkowo zwiększa interpretowalność i zrozumienie modelu.

Innym przykładem techniki lokalnej jest \textbf{SHAP} (SHapley Additive exPlanations)\cite{lundberg2017unified, SHAPmanip} wykorzystuje teorię gier do przypisywania znaczenia poszczególnym cechom wejściowym w procesie decyzyjnym modelu.
Metoda ta opiera się na szacowaniu wartości Shapleya, które określają, jaki wpływ ma każda cecha na przewidywaną wartość.
Dzięki tej technice możemy zidentyfikować, które cechy mają największy wpływ na wynik modelu i jakie są ich wzajemne zależności.

Teoria wartości Shapleya pochodzi z teorii gier kooperacyjnych i służą do podziału zysków między graczy (cechy w kontekście modelu) proporcjonalnie do ich wkładu w wynik.
Dla każdej cechy wartość Shapleya obliczana jest jako średni wkład tej cechy do wyniku modelu we wszystkich możliwych kombinacjach cech.

Metoda SHAP składa się z następujących kroków:
\begin{enumerate}
	\item \textbf{Wybór instancji}
	\item \textbf{Perturbacja danych} - Dla każdej cechy generowane są wszystkie możliwe podzbiory cech
	\item \textbf{Ocena wpływu cech} - Model jest uruchamiany dla każdego podzbioru cech, a różnice w wynikach są mierzone
	\item \textbf{Średnia wartości} - Wartości Shapleya dla danej cechy są obliczane jako średnia marginalnego wkładu tej cechy do wyniku modelu we wszystkich możliwych podzbiorach cech
\end{enumerate}

Obliczanie dokładnych wartości Shapleya może być obliczeniowo kosztowne, dlatego SHAP wykorzystuje różne techniki przyspieszanie obliczeń, takie jak aproksymacje i metody samplingowe.

SHAP to technika XAI, która pozwala na precyzyjne przypisywanie znaczenia poszczególnym cechom wejściowym.
W przypadku danych obrazowych, SHAP umożliwia identyfikację wpływu poszczególnych super-pikseli na predykcję modelu co jest kluczowe dla uzyskania wglądu w mechanizmy decyzyjne modelu.

\vspace{1cm}
Podsumowując, metody model-agnostic oferują elastyczne i uniwersalne podejście do wyjaśniania decyzji modeli, co czyni je atrakcyjnym narzędziem dla różnorodnych zastosowań.
Ich zdolność do pracy z różnymi rodzajami modeli oraz możliwość generowania lokalnych wyjaśnień pozwala na lepsze zrozumienie i interpretację działania modeli głębokiego uczenia.

\subsection*{Metody zależne od modelu}
Metody oparte na podejściu model-specific to techniki XAI, które są związane bezpośrednio z architekturą i działaniem konkretnego modelu.
Charakteryzują się one wysokim stopniem specjalizacji i dostosowania do konkretnych rodzajów modeli, co umożliwia uzyskanie bardziej szczegółowych i dokładnych wyjaśnień ich działania oraz mniejsze koszty obliczeniowe.

W odróżnieniu od metod niezależnych od modelu, metody te wykorzystują wewnętrzną strukturę i parametry modelu, aby dostarczyć wyjaśnień dotyczących podejmowanych decyzji.
Dzięki temu są w stanie dokładniej wskazać, które elementy modelu oraz które cechy mają kluczowy wpływ na wynik predykcji.

\textbf{GradCAM} \cite{Selvaraju_2019} (Gradient-weighted Class Activation Mapping) jest jedną z najbardziej popularnych metod model-specific, która umożliwia wizualizację obszarów obrazu, które najbardziej wpływają na decyzję klasyfikacji modelu.
Metoda ta wykorzystuje gradienty wsteczne, aby obliczyć istotność poszczególnych super-pikseli obrazu względem konkretnej klasy oraz warstwę konwolucyjną sieci CNN w celu wyznaczenia obszarów super-pikseli.
Dzięki temu możemy zidentyfikować, które obszary obrazu były decydujące dla klasyfikacji i w jaki sposób model dokonywał swoich predykcji.

GradCAM składa się z następujących kroków:
\begin{enumerate}
	\item \textbf{Forward pass} - przeprowadzamy forward pass przez model, aby uzyskać mapy cech i predykcje.
	\item \textbf{Obliczenie gradientów} - przeprowadzamy backward pass, aby uzyskać gradient klasy docelowej względem map cech.
	\item \textbf{Uśrednienie gradientów} - wykonujemy global average pooling, aby uzyskać wagi istotności.
	\item \textbf{Ważona suma map cech} - wykonujemy ważoną sumę map cech, wykorzystując wagi istotności.
	\item \textbf{ReLU} - zastosowanie funkcji ReLU do ważonej sumy, aby uzyskać ostateczną mapę cieplną.
	\item \textbf{Normalizacja} - Normalizujemy mapę cieplną do zakresu [0, 1].
  \item \textbf{Wizualizacja} - Stworzoną mapę cieplną nakładamy na obraz w celu wizualizacji.
\end{enumerate}

Poniżej znajduje się pseudo kod który wizualizuje kroki algorytmu GradCAM:

\begin{listing}
	\begin{minted}{python}
features = model.forward(image)
predictions = model.predict(features)
gradients = model.backward(class_idx, features)
weights = mean(gradients, axis=(0, 1))
cam = zeros(features.shape[1:3])
for i in range(len(weights)):
    gradcam += weights[i] * features[i]

gradcam = maximum(cam, 0)
gradcam = gradcam / max(gradcam)
    \end{minted}
	\caption{Pseudo kod GradCAM} \label{listing:gradcam}
\end{listing}

GradCAM dostarcza  wizualnych wyjaśnień, które są intuicyjne do interpretacji.
Dzięki mapom ciepła (heatmaps) generowanym przez tę metodę, możemy zobaczyć, które regiony obrazu najbardziej wpłynęły na predykcję modelu bez konieczności posiadania wiedzy eksperckiej.

\vspace{1cm}

Podsumowując, \textbf{Metody Model-specific} oferują wysoki poziom szczegółowości i precyzji w wyjaśnianiu działania konkretnych modeli, co czyni je atrakcyjnym narzędziem dla zrozumienia ich wewnętrznej logiki i mechanizmów decyzyjnych.
Ich dostosowanie do konkretnych architektur modeli pozwala na uzyskanie bardziej precyzyjnych i zrozumiałych wyjaśnień, co jest kluczowe w przypadku zastosowań wymagających wysokiego stopnia pewności i zaufania do modeli.
Dzięki temu metody takie jak GradCAM mogą nie tylko zwiększać zaufanie użytkowników do modeli, ale także prowadzić do ich dalszej optymalizacji i poprawy wyników.

\section*{Porównywanie metod XAI w klasyfikacji obrazów}
%Evaluating explainable artificial intelligence methods for multi-label deep learning classification tasks in remote sensing
%DHIS - Evaluation of Explainable Artificial Intelligence: SHAP, LIME, and CAM
%Ablation-CAM: Visual Explanations for Deep Convolutional Network via Gradient-free Localization
%CLEVER_XAI
Porównywanie metod wyjaśnialnej sztucznej inteligencji (XAI) w kontekście klasyfikacji obrazów jest kluczowe, aby zrozumieć, które techniki są najbardziej skuteczne i jakie mają zalety oraz ograniczenia.
W pracy 'Ablation-CAM'\cite{9093360} porównano GradCAM z autorskim rozwiązaniem.
Aby porównać obie metody zaproponowano dwa podejścia do porównywania metod XAI: oparte na modelu (model-based) oraz oparte na prawdzie (truth-based).
\begin{itemize}
	\item \textbf{Oparte na modelu} (model-based)
	\item \textbf{Oparte na prawdzie} (truth-based)
\end{itemize}

\subsection*{Oparte na modelu}
Metody oparte na modelu oceniają wyjaśnienia XAI poprzez analizie ich wpływu na wyniki modelu predykcyjnego.
Przykładowe metryki stosowane w tym podejściu:
\begin{itemize}
	\item \textbf{Średni spadek pewności i wyniku aktywacji}\cite{9093360} (Average drop in confidence and activation score) - ta metryka mierzy średni spadek pewności modelu (\ref{eq:average_drop}) w przypadku dostarczenia jedynie obszarów, które zostały zidentyfikowane jako ważne przez metodę XAI.
	      Jeśli metoda XAI poprawnie identyfikuje istotne cechy, powinno to prowadzić do niewielkiego spadku pewności predykcji modelu.
	      Metryka ta jest używana do oceny, jak skutecznie metoda XAI identyfikuje krytyczne cechy wpływające na decyzję modelu.
	      \begin{equation}
		      \text{Średni spadek} = \frac{1}{N} \sum_{i=1}^{N} \frac{\max(0,Y_i^c-O_i^c)}{T_i^c} \times 100
		      \label{eq:average_drop}
	      \end{equation}
	      Gdzie:
	      \begin{itemize}[label=]
		      \item $Y_i^c$ to wynik dla oryginalnego obrazu
		      \item $O_i^c$ to wynik dla samego obszaru wyjaśnienia
		      \item $N$ to liczba obrazów
	      \end{itemize}
	\item \textbf{Procentowy wzrost pewności i wyniku aktywacji}\cite{9093360} (Percent increase in confidence and activation score) - ta metryka mierzy procent przypadków wzrost pewności modelu (\ref{eq:rate_of-increase}), gdy tylko cechy zidentyfikowane jako ważne przez metodę XAI pozostają w obrazie, a wszystkie inne cechy są usunięte.
	      Jeśli metoda XAI prawidłowo identyfikuje istotne cechy, pozostawienie tylko tych cech powinno prowadzić do częstego wzrostu pewności predykcji modelu.
	      Ta metryka pozwala ocenić, w jakim stopniu metoda XAI może poprawić zrozumienie krytycznych cech przez model.
	      \begin{equation}
		      \text{Częstotliwość wzrostu pewności} =  \sum_{i=1}^{N} \frac{1_{Y_i^c<O_i^c}}{N} \times 100
		      \label{eq:rate_of-increase}
	      \end{equation}
	      Gdzie:
	      \begin{itemize}[label=]
		      \item $Y_i^c$ to wynik dla oryginalnego obrazu
		      \item $O_i^c$ to wynik dla samego obszaru wyjaśnienia
		      \item $N$ to liczba obrazów
	      \end{itemize}
\end{itemize}

\subsection*{Metody oparte na prawdzie}
Metody oparte na prawdzie oceniają wyjaśnienia XAI przez porównanie ich z zewnętrznymi, zdefiniowanymi prawdziwymi danymi. Jedną z głównych metryk w tym podejściu jest:
\begin{itemize}
	\item Intersection over Union (IoU)\cite{9093360, XAIevalCNN} - jest to miara pokrycia między wyjaśnieniami wygenerowanymi przez metodę XAI a zdefiniowanymi obszarami referencyjnymi w obrazie (\ref{eq:iou}), które są uważane za istotne (np. ręcznie oznaczone przez ekspertów).
	      IoU jest obliczane jako stosunek powierzchni przecięcia do powierzchni sumy obszarów wyjaśnień i referencji. Wysokie IoU wskazuje na dobrą zgodność wyjaśnień XAI z rzeczywistością, co oznacza, że metoda XAI skutecznie identyfikuje istotne cechy.
	      \begin{equation}
		      \text{IoU} = \frac{\text{I}}{\text{U}}
		      \label{eq:iou}
	      \end{equation}
	      Gdzie:
	      \begin{itemize}
		      \item I to powierzchnia przecięcia obszarów
		      \item U to powierzchnia sumy obszarów
	      \end{itemize}
\end{itemize}

%\section*{Zastosowania wyjaśnialnych modeli}
%
%Wyjaśnialne modele sztucznej inteligencji zwiększają transparentność i zrozumienie decyzji podejmowanych przez skomplikowane modele głębokiego uczenia.
%Poniżej przedstawiono praktyczne zastosowania XAI w różnych obszach.
%\subsection*{Medycyna}
%\begin{itemize}
%	\item Diagnostyka medyczna\cite{medicalXAIexample}
%	\item Planowanie leczenia\cite{medicalXAIexampletret}
%\end{itemize}
%\subsection*{Przemysł}
%\begin{itemize}
%	\item Predykcja utrzymania ruchu
%	\item Kontrola jakości\cite{qualityXAIexample}
%\end{itemize}
%\subsection*{Marketing i Reklama}
%\begin{itemize}
%	\item Marketing\cite{marketingXAIexample}
%\end{itemize}
%\subsection*{Automatyzacja i Robotyka}
%\begin{itemize}
%	\item Nawigacja i percepcja robotów\cite{lover2021explainable}
%\end{itemize}


% Metodyka [Przygotowanie środowiska, charakterystyka wybranych XAI, Algorytmy i miary jakości]
% !TEX encoding = UTF-8 Unicode 
% !TEX root = praca.tex

\chapter*{Metodologia}

\section*{Przygotowanie środowiska}

Do przeprowadzenia eksperymentów z wybranymi metodami XAI wymagane jest odpowiednie przygotowanie środowiska programistycznego.
W naszej pracy wykorzystujemy język Python oraz kilka popularnych bibliotek do uczenia maszynowego i przetwarzania obrazów.
Poniżej przedstawiamy krótki opis wykorzystywanych funkcji i bibliotek:
\begin{itemize}
	\item \textbf{Python} - język programowania, który jest szeroko stosowany w dziedzinie uczenia maszynowego.
	\item \textbf{TensorFlow} - biblioteka do uczenia maszynowego, która jest szeroko stosowana w dziedzinie uczenia maszynowego.
	\item \textbf{scikit-image} - biblioteka do przetwarzania obrazów, która zawiera wiele funkcji do przetwarzania obrazów.
	\item \textbf{LIME} - biblioteka do wyjaśniania modeli uczenia maszynowego, która jest szeroko stosowana w dziedzinie uczenia maszynowego.
	\item \textbf{SHAP} - biblioteka do wyjaśniania modeli uczenia maszynowego, która jest szeroko stosowana w dziedzinie uczenia maszynowego.
	\item \textbf{numpy} - biblioteka do obliczeń numerycznych, która jest szeroko stosowana w dziedzinie uczenia maszynowego.
	\item \textbf{matplotlib} - biblioteka do tworzenia wykresów, która jest szeroko stosowana w dziedzinie uczenia maszynowego.
\end{itemize}

\section*{Charakterystyka wybranych metod XAI}
W tej sekcji dokładniej omówimy trzy wybrane metody wyjaśnialnej sztucznej inteligencji (XAI), które zostaly wcześniej wymienione: LIME, SHAP oraz Grad-CAM.

\subsection*{LIME}
LIME jest techniką wyjaśnialnej sztucznej inteligencji, która generuje lokalne interpretacje modeli uczenia maszynowego.
Metoda ta jest agnostyczna względem modelu, co oznacza, że może być stosowana do różnych rodzajów modeli, niezależnie od ich architektury.
LIME działa poprzez tworzenie lokalnych modeli, które starają się naśladować oryginalny model dla konkretnego przypadku.
Jest to przydatne narzędzie do zrozumienia, dlaczego model dokonał konkretnej klasyfikacji dla danego przypadku testowego.

\subsection*{SHAP}
SHAP jest metodą opartą na teorii gier, która dostarcza globalnych interpretacji modeli uczenia maszynowego.
Wykorzystuje wartości Shapleya, aby obliczyć wpływ każdej cechy na predykcje modelu.
SHAP umożliwia zrozumienie, jak poszczególne cechy przyczyniają się do wyników modelu na poziomie globalnym.
Jest to przydatne narzędzie do identyfikowania najważniejszych cech wpływających na predykcje modelu.

\subsection*{Grad-CAM}
GradCAM to technika wizualizacji, która pozwala na lokalizowanie istotnych obszarów na obrazie, które przyczyniają się do konkretnej predykcji modelu.
Jest to przydatne narzędzie do zrozumienia, które obszary obrazu były decydujące dla decyzji modelu.
GradCAM wykorzystuje gradienty ostatniej warstwy sieci neuronowej w celu generowania map aktywacji klas, co umożliwia lokalizowanie obszarów najbardziej istotnych dla klasyfikacji.

\section*{Wybór hiperparametrów}

\section*{Algorytmy i miary jakości}

W tej sekcji opisujemy algorytmy oraz miary jakości użyte do oceny skuteczności metod wyjaśnialnej sztucznej inteligencji (XAI) w zadaniu klasyfikacji obrazów.
Możemy je podzielić na dwie kategorie, oparte na prawdzie oraz oparte na modelu.

\textbf{Miary oparte na prawdzie} są to miary, które oceniają skuteczność metod XAI na podstawie faktycznych wartości docelowych.

Natomiast \textbf{miary oparte na modelu} oceniają skuteczność metod XAI do wytłumaczania modelu.

\begin{enumerate}
	\item \textbf{IOU} (Intersection over Union): Algorytm IOU jest powszechnie stosowany w zadaniach segmentacji obrazów do oceny jakości detekcji obrazów.
	      Oblicza on stosunek powierzchni przecięcia dwóch obszarów do ich sumy.
	      W naszym kontekście, IOU może być używany do oceny zgodności obszarów wyznaczonych przez metody XAI z rzeczywistymi obiektami na obrazie.

	\item \textbf{Confidence change}: Miara ta ocenia zmianę pewności klasyfikacji na obrazach po zastosowaniu metod XAI.
	      Jest to różnica między pewnościami klasyfikacji na oryginalnych obrazach a pewnościami klasyfikacji na obrazach zmodyfikowanych z użyciem XAI.
	      Dzięki tej mierze możemy ocenić, czy metody XAI wpływają na zmianę pewności klasyfikacji.
	      Obszar wyznaczony przez metody XAI powinien minimalnie zmniejszyć pewność klasyfikacji lub ją zwiększyć.
	      Natomiast obszar poza wyznaczonym obszarem powinien drastycznie zmniejszyć pewność klasyfikacji.
\end{enumerate}

Poprzez zastosowanie tych algorytmów i miar jakości, będziemy w stanie ocenić skuteczność metod XAI w wyjaśnianiu klasyfikacji obrazów oraz ich wpływ na jakość klasyfikacji.



% Eksperymenty i wyniki [Analiza wyjaśnienia decyzji, porównanie zgodności wyjaśnień XAI, ocena skuteczności każdego z algorytmów]
% !TEX encoding = UTF-8 Unicode 
% !TEX root = praca.tex

\chapter*{Eksperymenty i wyniki}

\section*{Analiza porównawcza spójności wyjaśnień}
Porównanie spójności wyjaśnień między różnymi metodamii.
Porównanie spoójności w zależności od klasy obrazów

\section*{Analiza porównacza wyjaśnień}
Ocena wyjaśnień IOU, confidence score.
Ocena wyjaśnień w zależności od klasy.
Ocena wyjaśnień w zależności od wielkości obiektu(procent zajmowanego obszaru na obrazie)

\section*{Łączenie wyjaśnień różnych metod}
Porównanie spójności łączonych wyjaśnień (suma, średnia oraz część wspólna)
Ocena wyjaśnień dla połączonych wyjaśnień (suma, średnia oraz część wspólna).

\section*{Porównanie z innym zbiorem danych}

\section*{Porównanie z innymi modelami klasyfikacji obrazów}

\section*{Ocena poszczegóĺnych algorytmów}
Ocena poszczególnych algorytmów.
Porównanie wyników metod pod kątem ich zdolności do dostarczania zrozumiałych i precyzyjnych wyjaśnień.
Słabe i mocne strony każdej metody.
Czas wykonania eksperymentów oraz złożoność obliczeniowa



% Dyskusja [Omówienie wyników, Interpretacja metod, Wnioski]
% !TEX encoding = UTF-8 Unicode 
% !TEX root = praca.tex

\chapter*{Dyskusja}

\section*{Omówienie wyników}
Podsumowanie kluczowych informacji z sekcji "Eksperymenty i wyniki", w kontekście celów pracy.
Które metody najskuteczniejsze.
Potencjalne przyczyny różnic między metodami i ich wpływ na jakość wyjaśnień

\section*{Wnioski}



% Wnioski
% !TEX encoding = UTF-8 Unicode 
% !TEX root = praca.tex

\chapter*{Wnioski}

W przeprowadzonych badaniach skupiono się na analizie i porównaniu różnych metod wyjaśnialnej sztucznej inteligencji (XAI), w tym LIME, SHAP i GradCAM, pod kątem ich spójności, dokładności oraz wpływu na pewność modelu.
Przeanalizowano również, jak łączenie wyjaśnień tych metod wpływa na jakość i interpretowalność wyników.
Poniżej przedstawiono szczegółowe podsumowanie wyników oraz wnioski z przeprowadzonych badań.

\section*{Spójność obszarów wyjaśnień}

Analiza spójności obszarów wyjaśnień wykazała, że najlepszą spójność osiągano w przypadku połączenia GradCAM z LIME, podczas gdy najgorsze wyniki uzyskano dla połączenia LIME z SHAP.
Oznacza to, że GradCAM i LIME miały większą zgodność co do istotnych obszarów w obrazach, natomiast LIME i SHAP identyfikowały te obszary bardziej rozbieżnie.

\section*{Spójność w podziale na kategorie}

W podziale na kategorie również zaobserwowano, że połączenie GradCAM z SHAP charakteryzowało się najlepszą spójnością, a najgorsze wyniki uzyskano dla połączenia LIME z SHAP.
Najlepsze wyniki dla kategorii osiągnięto w przypadku kategorii ptak, natomiast najgorsze dla kategorii naczelny, co sugeruje, że GradCAM i SHAP lepiej współdziałały w identyfikacji istotnych cech u ptaków, podczas gdy LIME i SHAP miały trudności z naczelnikami.

\section*{Spójność w podziale na rozmiary}

Podział na rozmiary wykazał, że GradCAM z SHAP osiągały najlepszą spójność niezależnie od wielkości obrazu. Spójność ta zwiększała się proporcjonalnie do rozmiaru dla GradCAM z SHAP, podczas gdy dla GradCAM z LIME wzrost ten był nieproporcjonalny.
LIME z SHAP wykazywały podobne, choć nieco mniej spójne wyniki.
Oznacza to, że GradCAM i SHAP były bardziej stabilne i przewidywalne w identyfikacji istotnych obszarów w obrazach o różnych rozmiarach.

\section*{Analiza porównawcza metod}

Porównując metody XAI pod kątem wartości IoU, zmian pewności po pozostawieniu samego wyjaśnienia oraz zmian pewności po usunięciu wyjaśnienia, GradCAM okazał się najlepszy we wszystkich tych aspektach, natomiast LIME osiągało najgorsze wyniki.

Dla poszczególnych kategorii GradCAM zawsze miał najwyższe wartości IoU, podczas gdy LIME najniższe, z wyjątkiem kategorii Insekt, gdzie wyniki były zbliżone do SHAP.
Podobne wyniki zaobserwowano przy analizie zmian pewności, zarówno po pozostawieniu, jak i po usunięciu wyjaśnienia, gdzie GradCAM dominował nad LIME.

Połączenie wyjaśnień
\section*{Część wspólna}

Połączenie wyjaśnień przez część wspólną wykazało, że najlepsze wyniki osiągnięto dla połączenia GradCAM z LIME.
Jednakże, niezależnie od kombinacji metod, wyniki te były gorsze od wyników uzyskanych z poszczególnych metod.
Zmiany pewności modelu po pozostawieniu wyjaśnień były największe dla połączenia GradCAM z LIME, lecz nadal gorsze od podstawowych wyjaśnień.

\section*{Suma obszarów}

Połączenie wyjaśnień poprzez sumę obszarów przyniosło lepsze wyniki niż dla pojedynczych metod.
Najlepsze wyniki uzyskano dla połączenia GradCAM z LIME, natomiast najgorsze dla SHAP z LIME.
Wartości IoU były wyższe dla połączonych wyjaśnień, co sugeruje, że takie podejście może zwiększać dokładność identyfikacji istotnych obszarów.
Zmiana pewności modelu po pozostawieniu sumy obszarów wyjaśnień była największa dla połączenia wszystkich trzech metod, co sugeruje, że suma wyjaśnień może dostarczać bardziej kompleksowych informacji.
Jednak zmiana pewności po usunięciu tych obszarów również była największa dla tej kombinacji, co wskazuje na znaczną istotność tych obszarów dla modelu.

\section*{Wnioski końcowe}

Badania wykazały, że GradCAM jest najbardziej spójną i dokładną metodą wyjaśnień w porównaniu do LIME i SHAP. Łączenie wyjaśnień różnych metod może poprawić dokładność i spójność wyników, szczególnie gdy stosuje się sumę obszarów.
Jednakże, niezależnie od metody połączenia, wyniki te są zazwyczaj gorsze od wyników uzyskanych z poszczególnych metod.
W praktyce, wybór odpowiedniej metody XAI oraz podejście do łączenia wyjaśnień powinno być dostosowane do specyfiki problemu i wymagań interpretowalności modelu.


% !TEX encoding = UTF-8 Unicode 
% !TEX root = praca.tex

\chapter*{Podsumowanie}


Przeprowadzone badania skoncentrowały się na analizie i porównaniu metod wyjaśnialnej sztucznej inteligencji (XAI): LIME, SHAP oraz GradCAM. Celem było zrozumienie ich spójności, dokładności oraz wpływu na pewność modelu, a także ocena efektów łączenia wyjaśnień.

\section*{Główne osiągnięcia}

\begin{itemize}
	\item \textbf{Spójność wyjaśnień:} Najlepszą spójność osiągnęło połączenie GradCAM z LIME, natomiast najgorsze wyniki uzyskano dla LIME z SHAP.

	\item \textbf{Wielkość wyjaśnień:} GradCAM najlepiej dopasowuje wielkość obszarów wyjaśnień do rozmiaru obiektów na obrazach. SHAP wykazał brak korelacji między rozmiarem obszaru wyjaśnienia a faktycznym rozmiarem obiektu.

	\item \textbf{Porównanie metod XAI:} GradCAM osiągnął najlepsze wyniki metryki IoU, z wyjątkiem obiektów bardzo małych, gdzie lepiej poradził sobie LIME. LIME wykazał najmniejszy odsetek obszarów poza rzeczywistymi obiektami, co świadczy o jego precyzji w identyfikacji istotnych cech.

	\item \textbf{Łączenie wyjaśnień:} Połączenie GradCAM z LIME poprzez zsumowanie obszarów przyniosło najlepsze wyniki, sugerujące, że suma wyjaśnień może dostarczać bardziej kompleksowych informacji niż pojedyncze metody.
\end{itemize}

\section*{Znaczenie pracy}

Badania te są istotne dla wyjaśnialnej sztucznej inteligencji, pokazując, jak różne metody XAI mogą być używane w różnych kontekstach oraz jak ich kombinacja może poprawić dokładność i spójność wyjaśnień.
Pomimo że GradCAM osiągnął najlepsze wyniki, należy pamiętać, że jest to metoda specyficzna dla modeli zawierających warstwy konwolucyjne.
W związku z tym, GradCAM nie może być używany do wyjaśniania modeli, które nie posiadają warstw konwolucyjnych.

\section*{Perspektywy na przyszłość}

W przyszłości warto rozważyć:
\begin{itemize}
	\item Dalsze badania nad optymalizacją parametrów metod XAI w celu poprawy ich skuteczności.
	\item Zastosowanie innych metryk ewaluacyjnych dla bardziej szczegółowej analizy wyjaśnień.
	\item Zastosowanie bardziej zaawansowanych technik łączenia wyjaśnień w celu dalszej poprawy interpretowalności modeli.
	\item Analizę innych metod wyjaśnialnej sztucznej inteligencji.
\end{itemize}

Podsumowując, wyniki tych badań mogą przyczynić się do lepszego zrozumienia mechanizmów działania metod XAI oraz ich odpowiedniego stosowania w praktyce.


% Bibliography
\bibliographystyle{dyplom}
\bibliography{bibliography}

% Lists of figures, listings, tables
\listoffigures
\listoflistings
\listoftables

% Appendices - comment out if not applicable
\appendixpage
\appendix
\chapter{Dodatek}\label{app1}

\section*{Analiza porównawcza spójności wyjaśnień}

W tej sekcji przeprowadzono ocenę spójności wyjaśnień generowanych przez różne metody XAI: LIME, SHAP i GradCAM.
Celem analizy było zrozumienie, jak bardzo wyjaśnienia nakładają się na siebie oraz jak różne techniki identyfikują istotne cechy obrazu.

Wyniki wygenerowanych wyjaśnień zostały przedstawione za pomocą metryki IoU (Intersection over Union), która mierzy stopień pokrycia się regionów uznawanych za istotne przez różne metody.

\begin{figure}[h]
	\centering\includegraphics[width=.9\textwidth]{img/base_coherence}
	\caption{Spójność wyjaśnień}  \label{rys:base_coherence}
\end{figure}

\begin{table}[h]
	\centering
	\begin{tabular}{|c|c|}
		\hline
		\textbf{Metody XAI}     & \textbf{Średnie IoU} \\
		\hline
		\textbf{GradCAM i LIME} & 0.210750             \\
		\hline
		\textbf{GradCAM i SHAP} & 0.279382             \\
		\hline
		\textbf{LIME i SHAP}    & 0.113670             \\
		\hline
	\end{tabular}
	\caption{Średnie wartości IoU dla porównania spójności wyjaśnień}
	\label{tab:base_coherence}
\end{table}

Za pomocą wykresu pudełkowego (Rys \ref{rys:base_coherence}) przedstawione zostały wartości IoU dla porównania spójności wyjaśnień między różnymi metodami.
Natomiast Tabela \ref{tab:base_coherence} przedstawia porównanie spójności wyjaśnień między wybranymi metodami XAI, przy uzyciu średnich wartości IoU.

Wyniki pokazały, że największą spójnością wykazały się wyjaśnienia wygenerowane przez metody GradCAM i SHAP.
Natomiast najmniejszą spójność wykazały wyjaśnienia wygenerowane przez LIME i SHAP, co może wskazywać na różnice w podejściu tych metod do identyfikacji kluczowych cech.

Niska spójność między LIME i SHAP, pomimo wyższej spójności GradCAM i LIME oraz GradCAM i SHAP, może wskazywać na to, że obszary identyfikowane przez GradCAM są większe i bardziej ogólne.

GradCAM działa na poziomie cech wysokiego poziomu, które są identyfikowane na końcowych warstwach sieci neuronowej.
W związku z tym, GradCAM ma tendencję do zaznaczania większych regionów na obrazie, które zawierają kluczowe cechy wpływające na klasyfikację.
Dzięki temu wyjaśnienia generowane przez GradCAM są bardziej ogólne i obejmują szersze obszary obrazu.

LIME i SHAP z kolei, działają na bardziej szczegółowych poziomach.

W rezultacie, większe i bardziej ogólne regiony identyfikowane przez GradCAM mają większe szanse na nakładanie się z wyjaśnieniami LIME oraz SHAP.
Natomiast LIME i SHAP, ze względu na swoją szczegółowość, wykazują mniejszą spójność, ponieważ identyfikują bardziej precyzyjne i różniące się od siebie obszary.

\vspace{1cm}

W celu dokładniejszej analizy spójność wyjaśnień, wyniki podzielono ze względu na kategorie obrazów z bazy danych ImageNetS.
Analiza została przeprowadzona dla różnych kategorii obrazów.

\begin{figure}[h]
	\centering
	\begin{subfigure}[b]{0.3\textwidth}
		\includegraphics[width=.9\textwidth]{img/base_coherence_dog}
		\caption{\textbf{Dog}}  \label{}
	\end{subfigure}
	\begin{subfigure}[b]{0.3\textwidth}
		\centering\includegraphics[width=.9\textwidth]{img/base_coherence_bird}
		\caption{\textbf{Bird}}  \label{}
	\end{subfigure}
	\begin{subfigure}[b]{0.3\textwidth}
		\centering\includegraphics[width=.9\textwidth]{img/base_coherence_vehicle}
		\caption{\textbf{Vehicle}}  \label{}
	\end{subfigure}
	\begin{subfigure}[b]{0.3\textwidth}
		\centering\includegraphics[width=.9\textwidth]{img/base_coherence_reptile}
		\caption{\textbf{Reptile}}  \label{}
	\end{subfigure}
	\begin{subfigure}[b]{0.3\textwidth}
		\centering\includegraphics[width=.9\textwidth]{img/base_coherence_carnivore}
		\caption{\textbf{Carnivore}}  \label{}
	\end{subfigure}
	\begin{subfigure}[b]{0.3\textwidth}
		\centering\includegraphics[width=.9\textwidth]{img/base_coherence_insect}
		\caption{\textbf{Insect}}  \label{}
	\end{subfigure}
	\begin{subfigure}[b]{0.3\textwidth}
		\centering\includegraphics[width=.9\textwidth]{img/base_coherence_music}
		\caption{\textbf{Instrument}}  \label{}
	\end{subfigure}
	\begin{subfigure}[b]{0.3\textwidth}
		\centering\includegraphics[width=.9\textwidth]{img/base_coherence_primate}
		\caption{\textbf{Primate}}  \label{}
	\end{subfigure}
	\begin{subfigure}[b]{0.3\textwidth}
		\centering\includegraphics[width=.9\textwidth]{img/base_coherence_fish}
		\caption{\textbf{Fish}}  \label{}
	\end{subfigure}
	\caption{Spójność wyjaśnień dla różnych kategorii}
	\label{rys:coherence_category}
\end{figure}

\begin{table}[h]
	\centering
	\begin{tabular}{|c|c|c|c|}
		\hline
		\textbf{Kategoria}           & \textbf{GradCAM i LIME} & \textbf{GradCAM i SHAP} & \textbf{LIME i SHAP} \\
		\hline
		\textbf{Pies}                & 0.206546                & 0.277394                & 0.111613             \\
		\hline
		\textbf{Ptak}                & 0.247909                & 0.277716                & 0.117751             \\
		\hline
		\textbf{Pojazd na kołach}    & 0.204956                & 0.283621                & 0.118001             \\
		\hline
		\textbf{Gad}                 & 0.191600                & 0.287120                & 0.113589             \\
		\hline
		\textbf{Mięsorzerca}         & 0.196739                & 0.292880                & 0.110058             \\
		\hline
		\textbf{Insekt}              & 0.235127                & 0.273465                & 0.119031             \\
		\hline
		\textbf{Instrument muzyczny} & 0.208132                & 0.253540                & 0.109638             \\
		\hline
		\textbf{Naczelny}            & 0.195163                & 0.283483                & 0.105888             \\
		\hline
		\textbf{Ryba}                & 0.210879                & 0.285220                & 0.117462             \\
		\hline
	\end{tabular}
	\caption{Średnie wartości IoU dla różnych kategorii}
	\label{tab:base_coherence_categories}
\end{table}

Dane zostały przedstawione w ten sam sposób jak wcześniej, za pomocą wykresu (Rys \ref{rys:coherence_category}) oraz tabeli (Tabela \ref{tab:base_coherence_categories}) przedstawiającej średnie wartości IoU w celu prównania spójności w zależności od kategorii obrazu.

Spójność LIME i SHAP jest podobna dla wszystkich kategorii obrazów, przy czym najgorsza jest dla kategorii Naczelny, natomiast najlepsza dla kategorii Insect.
Spójność między tymi dwoma metodami jest najgorsza niezależnie od kategorii.

Spójność GradCAM i LIME jest różna w zależności od kategorii obrazów, przy czym najgorsza jest dla kategorii Gad, natomiast najlepsza dla kategorii Ptak i Insekt.
Spójność między tymi dwiema metodami nigdy nie jest najlepsza ani najgorsza niezależnie od kategorii.

Spójność GradCAM i SHAP jest różna w zależności od kategorii obrazów, przy czym najgorsza jest dla kategorii Instrument muzyczny, natomiast najlepsza dla Mięsorzerca.
Spójność miedzy tymi dwiema metodami zawsze jest najlepsza nie zależnie od kategorii.

\vspace{1cm}
W celu zweryfikowania czy wielkość wyjaśnień miała znaczenie przeanalizowano spójność wyjaśnień podzieloną ze względu na wielkość obiektów na obrazie.

\begin{figure}[h]
	\centering
	\begin{subfigure}[b]{0.3\textwidth}
		\includegraphics[width=1\textwidth]{img/base_coherence_size_S}
		\caption{Mały obiekt}  \label{}
	\end{subfigure}
	\begin{subfigure}[b]{0.3\textwidth}
		\centering\includegraphics[width=1\textwidth]{img/base_coherence_size_M}
		\caption{Średni obiekt}  \label{}
	\end{subfigure}
	\begin{subfigure}[b]{0.3\textwidth}
		\centering\includegraphics[width=1\textwidth]{img/base_coherence_size_L}
		\caption{Duży obiekt}  \label{}
	\end{subfigure}
	\caption{Spójność wyjaśnień dla różnych rozmiarów obiektów}
	\label{rys:coherence_size}
\end{figure}

\begin{table}[h]
	\centering
	\begin{tabular}{|c|c|c|c|}
		\hline
		\textbf{Rozmiar} & \textbf{GradCAM vs LIME} & \textbf{GradCAM vs SHAP} & \textbf{LIME vs SHAP} \\
		\hline
		\textbf{Mały}    & 0.253597                 & 0.250014                 & 0.117247              \\
		\hline
		\textbf{Średni}  & 0.204279                 & 0.282094                 & 0.114313              \\
		\hline
		\textbf{Duży}    & 0.172506                 & 0.307551                 & 0.109135              \\
		\hline
	\end{tabular}
	\caption{Średnie wartości IoU dla różnych rozmiarów}
	\label{tab:base_coherence_size}
\end{table}

Tak jak w poprzednich przypadkach wyniki przedstawiono na wykresie (Rys \ref{rys:coherence_size}) oraz tabeli (Tabela \ref{tab:base_coherence_size}) przedstawiającej średnie wartości IoU dla porównania spójności wyjaśnień w zależności od rozmiaru obiektu na obrazie.

Wyniki:
\begin{itemize}
	\item \textbf{Obiekty małe} - Największa spójność wyjaśnień występuje między metodami GradCAM i LIME oraz GradCAM i SHAP.
	      Spójność między LIME i SHAP jest znacznie niższa tak jak w poprzednich analizach.
	\item \textbf{Obiekty średnie} - GradCAM i SHAP ponownie wykazują największą spójność, następnie GradCAM i LIME, a najniższą spójność odnotowano między LIME i SHAP.
	\item \textbf{Obiekty duże} - Spójność wyjaśnień między GradCAM i SHAP jest najwyższa. GradCAM i LIME mają niższą spójność, a najniższą spójność obserwuje się między LIME i SHAP.
\end{itemize}

Dodatkowo zauważono, że spójność między GradCAM i LIME była odwrotnie proporcjonalna do wielkości obiektu.
Im mniejeszy obiekt, tym większa była spójność między tymi metodami.
Podobny trend zaobserwowano w przypadku spójności między LIME i SHAP, choć wpływ wielkości obiektu był mniejszy.
Natomiast spójność między GradCAM i SHAP była proporcjonalnie zależna od wielkośći.
Im większy obiekt, tym większa była spójność między tymi metodami.

Warto zauważyć, że metody takie jak LIME i SHAP są wrażliwe na dobór parametrów.
Te zależności od parametrów mogą wpływać na stabilność i spójność wyjaśnień generowanych przez te techniki.
Co może częściowo może tłumaczyć podobne wyniki między LIME i SHAP niezależnie od rozmiaru obiektu.



\begin{figure}[h]
	\centering
	\begin{subfigure}[b]{0.3\textwidth}
		\includegraphics[width=.9\textwidth]{img/examples/appendix/n03272562_40246_gradcam}
		\caption{GradCAM: 34.693878\%, rozmiar obiektu: 34.689892\%}
	\end{subfigure}
	\begin{subfigure}[b]{0.3\textwidth}
		\includegraphics[width=.9\textwidth]{img/examples/appendix/n02526121_01747_lime}
		\caption{LIME: 7.001355\%, rozmiar obiektu: 7.009327\%}
	\end{subfigure}
	\begin{subfigure}[b]{0.3\textwidth}
		\includegraphics[width=.9\textwidth]{img/examples/appendix/n02493793_39302_shap}
		\caption{SHAP: 13.281250\%, rozmiar obiektu: 13.299187\%}
	\end{subfigure}
	\caption{Przykład rozmiaru wyjaśniania bliskiego rzeczywistym rozmiarom obiektu}
	\label{}
\end{figure}

\begin{figure}[h]
	\centering
	\begin{subfigure}[b]{0.3\textwidth}
		\includegraphics[width=.9\textwidth]{img/examples/appendix/n02493793_32964_gradcam}
		\caption{GradCAM: 18.367347\%, rozmiar obiektu: 99.583466\%}
	\end{subfigure}
	\begin{subfigure}[b]{0.3\textwidth}
		\includegraphics[width=.9\textwidth]{img/examples/appendix/n02488291_05090_lime}
		\caption{LIME: 5.781649\%, rozmiar obiektu: 98.654735\%}
	\end{subfigure}
	\begin{subfigure}[b]{0.3\textwidth}
		\includegraphics[width=.9\textwidth]{img/examples/appendix/n02129165_33659_shap}
		\caption{SHAP: 3.125000\%, rozmiar obiektu: 97.193878\%}
	\end{subfigure}
	\caption{Przykład rozmiaru wyjaśniania dalekiego rzeczywistym rozmiarom obiektu}
	\label{}
\end{figure}



\begin{figure}[h]
	\centering
	\begin{subfigure}[b]{0.3\textwidth}
		\includegraphics[width=.9\textwidth]{img/examples/appendix/n02086646_12074_gradcam}
		\caption{GradCAM IoU=0.439218}
	\end{subfigure}
	\begin{subfigure}[b]{0.3\textwidth}
		\includegraphics[width=.9\textwidth]{img/examples/appendix/n01560419_47474_lime}
		\caption{LIME IoU=0.176749}
	\end{subfigure}
	\begin{subfigure}[b]{0.3\textwidth}
		\includegraphics[width=.9\textwidth]{img/examples/appendix/n03670208_41512_shap}
		\caption{SHAP IoU=0.118568}
	\end{subfigure}
	\caption{Przykłady wyjaśnień dla których IoU jest bliskie ich średniej}
	\label{}
\end{figure}

\begin{figure}[h]
	\centering
	\begin{subfigure}[b]{0.3\textwidth}
		\includegraphics[width=.9\textwidth]{img/examples/appendix/n01695060_35102_gradcam}
		\caption{GradCAM IoU=0.988630}
	\end{subfigure}
	\begin{subfigure}[b]{0.3\textwidth}
		\includegraphics[width=.9\textwidth]{img/examples/appendix/n02279972_47456_lime}
		\caption{LIME IoU=0.619109}
	\end{subfigure}
	\begin{subfigure}[b]{0.3\textwidth}
		\includegraphics[width=.9\textwidth]{img/examples/appendix/n04037443_42773_shap}
		\caption{SHAP IoU=0.486928}
	\end{subfigure}
	\caption{Przykłady wyjaśnień dla których IoU jest większe niż normalnie}
\end{figure}

\begin{figure}[h]
	\centering
	\begin{subfigure}[b]{0.3\textwidth}
		\includegraphics[width=.9\textwidth]{img/examples/appendix/n02119022_08938_gradcam}
		\caption{GradCAM - procent 36.305589\%}
	\end{subfigure}
	\begin{subfigure}[b]{0.3\textwidth}
		\includegraphics[width=.9\textwidth]{img/examples/appendix/n01739381_08888_lime}
		\caption{LIME - procent 28.515625\%}
	\end{subfigure}
	\begin{subfigure}[b]{0.3\textwidth}
		\includegraphics[width=.9\textwidth]{img/examples/appendix/n02120505_19480_shap}
		\caption{SHAP - procent 45.586735\%}
	\end{subfigure}
	\caption{Przykłady wyjaśnień dla których procent nieprawidłowo zaznaczonych obszarów jest na średnim poziomie}
	\label{}
\end{figure}

\begin{figure}[h]
	\centering
	\begin{subfigure}[b]{0.3\textwidth}
		\includegraphics[width=.9\textwidth]{img/examples/appendix/n03495258_37094_gradcam}
		\caption{GradCAM - procent 100\%}
	\end{subfigure}
	\begin{subfigure}[b]{0.3\textwidth}
		\includegraphics[width=.9\textwidth]{img/examples/appendix/n02177972_27717_lime}
		\caption{LIME - procent 100\%}
	\end{subfigure}
	\begin{subfigure}[b]{0.3\textwidth}
		\includegraphics[width=.9\textwidth]{img/examples/appendix/n01756291_07991_shap}
		\caption{SHAP - procent 100\%}
	\end{subfigure}
	\caption{Przykłady wyjaśnień dla których procent nieprawidłowo zaznaczonych obszarów jest na bardzo wysokim poziomie}
	\label{}
\end{figure}



\end{document}
